% !Mode:: "TeX:UTF-8"
% !TEX builder = LATEXMK
% !TEX program = xelatex

% TODO disable draft mode and enable anon if necessary
\documentclass[master,twoside,nocpsupervisor]{style/zjuthesis}
% \documentclass[anon,master,twoside,nocpsupervisor]{style/zjuthesis}

% 插图路径设置,图片放在figures 文件夹下。一般来说论文的插图比较多,通常按章节存
% 放,因此可以在以下命令中在按章节添加存放图片的文件夹路径。如以下这个路径中 ./
% 代表当前main.tex所在的目录,就是一般所说的当前文件夹;figures 文件夹就是子文件
% 夹,存放正文及附录中要用到的所有的图片,在figures 文件夹中的子文件夹就是存放各
% 个章节图片的文件夹,一般命名与相应章节的名字相同,如intro 章节用到的图片全放在
% 了intro 这个子文件夹下。
% \graphicspath{%
%     {./figures/intro/}%
% }
\input{thesis/symbols.tex}
\input{thesis/figures.tex}

% 论文中文标题
\title{研究生毕业论文总结报告\LaTeX{}模板}
% 论文英文标题
\englishtitle{\LaTeX{} Template for Thesis}
% 作者,就是你的名字
\author{郝仁}
% 分类号
\classification{TM863}
% 单位代码
\serialnumber{10335}
% 密级
\secretlevel{公开}
% 学号
\studentnumber{54321}
% 指导教师
\supervisor{渡鸦12345}
% 合作导师,如果没有合作导师,就在\documentclass选项栏中加上"nocpsupervisor"。
\cpsupervisor{无}
% 专业名称
\major{时空管理局}
% 研究方向
\research{世界和平}
% 所在学院
\institute{希灵帝国}
% 提交日期
\submitdate{\today}

% 中文题名页
\reviewerA{}
\reviewerB{}
\reviewerC{}
\reviewerD{}
\reviewerE{}
\chairperson{}
\commissionerA{}
\commissionerB{}
\commissionerC{}
\commissionerD{}
\commissionerE{}
\defencedate{\today}

% 英文题名页
\enreviewerA{}
\enreviewerB{}
\enreviewerC{}
\enreviewerD{}
\enreviewerE{}
\enchairperson{}
\encommissionerA{}
\encommissionerB{}
\encommissionerC{}
\encommissionerD{}
\encommissionerE{}
\eendefencedate{\today}

\begin{document}

\maketitle
% \ZJUmakecover
% \ZJUmakeCNtitlepage
% \ZJUmakeENtitlepage

\frontmatter
% !Mode:: "TeX:UTF-8"
% !TEX root = ../thesis.tex

\ifanon
\else
\chapter{致\texorpdfstring{\ZJUspace}{}谢}
岁月如梭,转眼间,为时两年半的摸鱼研究生生涯即将结束。
在此期间,我认识了许多关心、爱护、帮助我的人们,他们激励着我在求学路上奋勇向前。
浙江大学学习风气优良、科研氛围严谨、校园生活充实。
正是在此种环境与人们的影响下,我才能顺利进行科研与生活。
值此毕业论文完成之际,谨向所有关心、爱护、帮助我的人们表示最诚挚的感谢与最美好的祝愿。

首先,需要感谢我的导师老咸鱼教授。

其次,要感谢实验室的各位鱼类同学。

最后,感谢父母在我求学生涯中一如既往地理解、支持与鼓励我。
这份亲情是我求学最大的动力,激励我能更加坚定地走向前方、拥抱梦想。

\vspace{2cm}
\hfill
\begin{minipage}{14em}
\begin{center}
于浙江大学\quad 2020年1月10日\\
咸鱼
\end{center}
\end{minipage}
\fi

% !Mode:: "TeX:UTF-8"
% !TEX root = ../thesis.tex

% 定义中文摘要和关键字
\begin{cabstract}
基于自监督学习的摸鱼神经网络
\end{cabstract}

\ckeywords{摸鱼}

% !Mode:: "TeX:UTF-8"
% !TEX root = ../thesis.tex

% 定义英文摘要和关键字
\begin{eabstract}
SlackNet: How to Slack Off Happily via Self-Supervised Learning
\end{eabstract}

\ekeywords{slack off}


% 正文目录:
\tableofcontents
% 插图目录:
% \listoffigures
% 表格目录:
% \listoftables
% \include{contents/denotation}

\mainmatter

% !Mode:: "TeX:UTF-8"
% !TEX root = ../thesis.tex

\chapter{绪论}
\label{cha:introduction}

\section{研究背景}
\label{sec:background}
% section 研究背景 (end)

\section{国内外研究现状}
\label{sec:related_works}
% section 国内外研究现状 (end)

\section{本论文研究内容}
\label{sec:thesis_target}
% section 本论文研究内容 (end)

\section{本论文结构安排}
\label{sec:thesis_structure}
% section 本论文结构安排 (end)
% chapter 绪论 (end)


\backmatter

\bibliography{thesis}
% \nocite{*} % to show the entire references, annotate it if need.

% \appendix
% \include{thesis/appendixA}
% \include{thesis/appendixB}
\end{document}
